\documentclass{m1interim}

% ---- 追加パッケージ(必要に応じて) ----
\usepackage{url}

% ---- タイトル情報 ----
\jptitle{日本語タイトル}
\entitle{English Title Goes Here}
\department{システム理工学専攻}
\studentid{MF23000}
\studentname{芝浦 太郎}
\labname{○○○○○○○○研究室}
\supervisor{芝浦 工大郎}

\begin{document}
\maketitle

% ==============================
% 1. 背景
% ==============================
\section{背景}

研究背景,意義,研究目的などをここに記述する\cite{ref1}.
本文は10ptで記述され,2段組構成となっている.
1段1行あたり約20文字を目安とする\cite{ref2}.

% ==============================
% 2. 方法
% ==============================
\section{方法}

研究方法についてここに記述する.

% ==============================
% 3. 結果
% ==============================
\section{結果}

研究結果についてここに記述する.
図表の挿入も可能である.各図表にはタイトルを必ずつけること.

% 図の挿入例:
% \begin{figure}[t]
%   \centering
%   \includegraphics[width=\columnwidth]{fig_example.pdf}
%   \caption{図のキャプション}
%   \label{fig:example}
% \end{figure}

% 表の挿入例:
% \begin{table}[t]
%   \centering
%   \caption{表のキャプション}
%   \label{tab:example}
%   \begin{tabular}{cc}
%     \hline
%     項目A & 項目B \\
%     \hline
%     1 & 2 \\
%     \hline
%   \end{tabular}
% \end{table}

% ==============================
% 4. 考察
% ==============================
\section{考察}

考察をここに記述する.
本論となる2〜4は,研究分野により,異なる構成でも可.

% ==============================
% 5. 結論
% ==============================
\section{結論}

まとめ,今後の課題などをここに記述する.

% ==============================
% 参考文献
% ==============================
\bibliographystyle{unsrt}
\bibliography{refs}

\end{document}
